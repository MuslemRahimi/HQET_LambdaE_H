\documentclass[openright,twoside,12pt,a4paper,final]{article}
\usepackage{a4wide}
\usepackage[utf8]{inputenc}
\usepackage{blindtext}
\usepackage{amsmath}
\usepackage{amsfonts}
\usepackage{amssymb}
\usepackage{makeidx}
\usepackage{graphicx}
\usepackage{float}
\usepackage{titlesec}
\usepackage{listings}
\usepackage{hyperref}
\usepackage[english]{babel}
%\usepackage{classicthesis}
\usepackage{xcolor}
\usepackage{tikz}
\usepackage{pdfpages}
\usepackage[ngerman]{datetime}
\usepackage{bm}
\usepackage{subcaption}
\usepackage{changepage}
\usepackage{braket}
\usepackage{slashed}
\usepackage{simplewick}

\begin{document}
\section{Reproduce Eq.(25)}
	First we Taylor expand the following matrix element:
	\begin{align}
	    \bra{0} \bar{q}(0) \Gamma_{1} P_{+} \Gamma_{2} \,q(x) \ket{0} &= \bra{0} \bar{q}(0) \Gamma_{1} P_{+} \Gamma_{2} q(0) \ket{0} + x^{\mu} \bra{0} \bar{q}(0) \Gamma_{1} P_{+} \Gamma_{2} D_{\mu} q(0) \ket{0} \nonumber \\
	    & +\frac{1}{2} x^{\mu} x^{\nu} \bra{0} \bar{q}(0) \Gamma_{1} P_{+} \Gamma_{2} D_{\mu} D_{\nu} q(0) \ket{0} + \cdots
	    \label{eq::matrix25}
	\end{align}
    First term in Eq.(\ref{eq::matrix25}) corresponds to the quark-antiquark condensate. 
    \begin{align}
         \bra{0} \bar{q}^{i}_{\alpha}(0) \Gamma_{1,\alpha \beta} P_{+, \beta\gamma} \Gamma_{2, \gamma \delta} \,q^{j}_{\delta}(0) \ket{0} \delta^{ij} &= A \delta^{ij} \delta_{\alpha \delta} \nonumber \\
         \Leftrightarrow \bra{0} \bar{q}^{i}_{\alpha}(0) \Gamma_{1,\alpha \beta} P_{+, \beta\gamma} \Gamma_{2, \gamma \delta} \,q^{j}_{\delta}(0) \ket{0} \delta^{ij} \delta^{ij}\delta_{\alpha \delta} &= A \delta^{ij} \delta^{ij} \delta_{\alpha \delta} \delta_{\alpha \delta}\nonumber \\
          \Leftrightarrow \text{Tr}[\Gamma_{1} P_{+} \Gamma_{2}] \braket{\bar{q} q} N_{c}  &= 4 A N_{c} \nonumber \\
          \Rightarrow A = \frac{1}{4} \text{Tr}[\Gamma_{1} P_{+} \Gamma_{2}] \braket{\bar{q} q}
         \label{eq::quark-antiquark}
    \end{align}
    where $(i,j)$ are color indices and $(\alpha, \beta,\gamma,\delta)$ are spinor indices. In the second line of Eq.(\ref{eq::quark-antiquark}) we multiplied on both sides $\delta^{ij} \delta_{\alpha \delta}$.
    
    The second term in Eq.(\ref{eq::matrix25}) does not contribute since according to Dirac's equation we can rewrite the covariant derivative as:
    \begin{align}
        \slashed{D} q = - i m_{q} q.
    \end{align}
    In HQET we assume $m_{q} = 0$ for light quarks.
    
    The third term in Eq.(\ref{eq::matrix25}) corresponds to the quark-antiquark-gluon condensate.
    \begin{align}
        \bra{0}\bar{q}^{i}_{\alpha}(0) \Gamma_{1, \alpha \beta} P_{+, \beta \gamma} \Gamma_{2, \gamma \delta} D_{\mu} D_{\nu} q^{j}_{\delta}(0) \ket{0}  &= C_{1} \delta^{ij} \delta_{\alpha \delta} \, g_{\mu \nu} + C_{2}  \delta^{ij} \delta_{\alpha \delta} \sigma_{\mu \nu},
        \label{eq::quark-antiquark-gluon1}
    \end{align}
    where we are only interested in the symmetric part. Now to compute $C_{1}$ we rewrite again Eq.(\ref{eq::quark-antiquark-gluon1}) using translation invariance as:
    \begin{align}
        \bra{0}\bar{q}^{i}_{\alpha}(0) \Gamma_{1, \alpha \beta} P_{+, \beta \gamma} \Gamma_{2, \gamma \delta} D_{\mu} D_{\nu} q^{j}_{\delta}(0) \ket{0} \delta^{ij} &= \bra{0}\bar{q}^{i}_{\alpha}(0) \Gamma_{1, \alpha \beta} P_{+, \beta \gamma} \Gamma_{2, \gamma \delta} G_{\mu \nu} q^{j}_{\delta}(0) \ket{0} \delta^{ij} \nonumber \\
        & +\bra{0}\bar{q}^{i}_{\alpha}(0) \Gamma_{1, \alpha \beta} P_{+, \beta \gamma} \Gamma_{2, \gamma \delta} D_{\nu} D_{\mu} q^{j}_{\delta}(0) \ket{0} \delta^{ij}.
    \end{align}
    Now by using Lorentz covariance we can re-express the first term:
    \begin{align}
        \bra{0}\bar{q}^{i}_{\alpha}(0) \Gamma_{1, \alpha \beta} P_{+, \beta \gamma} \Gamma_{2, \gamma \delta} G_{\mu \nu} q^{j}_{\delta}(0) \ket{0}  &= E \delta^{ij} \delta_{\alpha \delta} \sigma_{\mu\nu} \nonumber \\
        \Leftrightarrow \bra{0}\bar{q}^{i}_{\alpha}(0) \Gamma_{1, \alpha \beta} P_{+, \beta \gamma} \Gamma_{2, \gamma \delta} G_{\mu \nu} q^{j}_{\delta}(0) \ket{0} \delta^{ij} \delta_{\alpha \delta} \sigma^{\mu \nu} &= E \delta^{ij} \delta_{\alpha \delta} \sigma_{\mu\nu} \delta^{ij} \delta_{\alpha \delta} \sigma^{\mu\nu} \nonumber \\
        \Rightarrow E = \frac{1}{12 d(d-1)} \text{Tr}[\Gamma_{1} P_{+} \Gamma_{2}]\braket{\bar{q} \, G \cdot \sigma q},
    \end{align}
    where we multiplied both sides with $\delta^{ij} \delta_{\alpha \delta} \sigma^{\mu\nu}$ and used $\sigma_{\mu \nu} \sigma^{\mu \nu} = d(d-1)$.
    
    \newpage
    \textcolor{red}{To obtain $C_{1}$ is not written here very clear but will do it very soon.}
    Hence, the result for $C_{1}$ is:
    \begin{align}
        C_{1} &= i \left(\frac{1-d}{2} \right) E.
    \end{align}
    
    Therefore, we obtain for the hadronic matrix element:
    \begin{align}
        \frac{1}{2} x^{\mu} x^{\nu} \bra{0}\bar{q}^{i}_{\alpha}(0) \Gamma_{1, \alpha \beta} P_{+, \beta \gamma} \Gamma_{2, \gamma \delta} D_{\mu} D_{\nu} q^{j}_{\delta}(0) \ket{0} \delta^{ij} &= C_{1} \delta^{ij} \delta^{ij} \delta_{\alpha \delta} g_{\mu \nu} \nonumber \\
        &=  \frac{1}{2} x^{\mu} x^{\nu} i \left(\frac{1-d}{2} \right) \frac{1}{12 d (d-1)} g_{\mu \nu} \nonumber \\
        & \times \text{Tr}[\Gamma_{3} P_{+} \Gamma_{2}] \braket{\bar{q} \, G \cdot \sigma q} \nonumber \\
        &= \frac{1}{4} \text{Tr}[\Gamma_{3} P_{+} \Gamma_{2}] \left(\frac{-i}{16} \braket{\bar{q} \, G \cdot \sigma q} \right).
    \end{align}
    \textcolor{red}{Note that, we have a $(-i)$ too much. In the literature the $-i$ also appears so I have to think about it.}
    \end{document}