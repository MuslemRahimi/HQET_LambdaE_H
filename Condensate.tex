\documentclass[openright,twoside,12pt,a4paper,final]{article}
\usepackage{a4wide}
\usepackage[utf8]{inputenc}
\usepackage{blindtext}
\usepackage{amsmath}
\usepackage{amsfonts}
\usepackage{amssymb}
\usepackage{makeidx}
\usepackage{graphicx}
\usepackage{float}
\usepackage{titlesec}
\usepackage{listings}
\usepackage{hyperref}
\usepackage[english]{babel}
%\usepackage{classicthesis}
\usepackage{xcolor}
\usepackage{tikz}
\usepackage{pdfpages}
\usepackage[ngerman]{datetime}
\usepackage{bm}
\usepackage{subcaption}
\usepackage{changepage}
\usepackage{braket}
\usepackage{slashed}
\usepackage{simplewick}

\begin{document}
\section{Reproduce Eq.(25)}
	First we Taylor expand the following matrix element:
	\begin{align}
	    \bra{0} \bar{q}(0) \Gamma_{1} P_{+} \Gamma_{2} \,q(x) \ket{0} &= \bra{0} \bar{q}(0) \Gamma_{1} P_{+} \Gamma_{2} q(0) \ket{0} + x^{\mu} \bra{0} \bar{q}(0) \Gamma_{1} P_{+} \Gamma_{2} D_{\mu} q(0) \ket{0} \nonumber \\
	    & +\frac{1}{2} x^{\mu} x^{\nu} \bra{0} \bar{q}(0) \Gamma_{1} P_{+} \Gamma_{2} D_{\mu} D_{\nu} q(0) \ket{0} + \cdots
	    \label{eq::matrix25}
	\end{align}
    First term in Eq.(\ref{eq::matrix25}) corresponds to the quark-antiquark condensate. 
    \begin{align}
         \bra{0} \bar{q}^{i}_{\alpha}(0) \Gamma_{1,\alpha \beta} P_{+, \beta\gamma} \Gamma_{2, \gamma \delta} \,q^{j}_{\delta}(0) \ket{0} &= A \delta^{ij} \delta_{\alpha \delta} \Gamma_{1,\alpha \beta} P_{+, \beta\gamma} \Gamma_{2, \gamma \delta} \nonumber \\
         \Leftrightarrow \bra{0} \bar{q}^{i}_{\alpha}(0) \,q^{j}_{\delta}(0) \ket{0} \delta^{ij} \delta_{\alpha \delta} &= A \delta^{ij} \delta^{ij} \delta_{\alpha \delta} \delta_{\alpha \delta}\nonumber \\
          \Leftrightarrow \braket{\bar{q} q}   &= 4 A N_{c} \nonumber \\
          \Rightarrow A = \frac{1}{4  N_c} \braket{\bar{q} q}
         \label{eq::quark-antiquark}
    \end{align}
    where $(i,j)$ are color indices and $(\alpha, \beta,\gamma,\delta)$ are spinor indices. In the second line of Eq.(\ref{eq::quark-antiquark}) we multiplied on both sides $\delta^{ij} \delta_{\alpha \delta}$. Combining this result with \eqref{eq::matrix25}, we find
    \begin{align}
	    \bra{0} \bar{q}^{i}_{\alpha}(0) \Gamma_{1,\alpha \beta} P_{+, \beta\gamma} \Gamma_{2, \gamma \delta} \,q^{j}_{\delta}(0) \ket{0} =& \; \Gamma_{1,\alpha \beta} P_{+, \beta\gamma} \Gamma_{2, \gamma \delta} \cdot \bra{0} \bar{q}^{i}_{\alpha}(0) \,q^{j}_{\delta}(0) \ket{0} = \Gamma_{1,\alpha \beta} P_{+, \beta\gamma} \Gamma_{2, \gamma \delta} \cdot A \delta^{ij} \delta_{\alpha \delta} \nonumber \\ =& \; \Gamma_{1,\alpha \beta} P_{+, \beta\gamma} \Gamma_{2, \gamma \delta} \cdot \frac{1}{4  N_c} \braket{\bar{q} q} \delta^{ij} \delta_{\alpha \delta} \nonumber \\ =& \frac{1}{4  N_c} \cdot \text{Tr}[\Gamma_{1} P_{+} \Gamma_{2}] \braket{\bar{q} q} \delta^{ij}
    \end{align}
    Here, there will be an additional $\delta^{ij}$ due to the summation over the color indices, which cancels the factor $\frac{1}{N_c}$. 
    
    The second term in Eq.(\ref{eq::matrix25}) does not contribute since according to Dirac's equation we can rewrite the covariant derivative as:
    \begin{align}
        \slashed{D} q = - i m_{q} q.
    \end{align}
    In HQET we assume $m_{q} = 0$ for light quarks.
    
    Before we consider the third term in more detail, we take a closer look at the following matrix element:
    \begin{align}
	    \bra{0} \bar{q}_{\alpha}^i(0)g_s G_{\mu \nu}(0) q_{\delta}^j(0) \ket{0} =& \; E \cdot \delta^{ij} \cdot (\sigma_{\mu \nu})_{\delta \alpha} \nonumber \\ \Leftrightarrow \bra{0} \bar{q}_{\alpha}^i(0)g_s G_{\mu \nu}(0) q_{\delta}^j(0) \ket{0} \delta^{ij} (\sigma^{\mu \nu})_{\alpha \delta} =& \; E \cdot N_c \cdot \mathrm{Tr}[\sigma^{\mu \nu} \sigma_{\mu \nu}] \nonumber \\ \Leftrightarrow E = \bra{0} \bar{q} g_s \sigma \cdot G q \ket{0} \cdot \frac{1}{4 N_c d (d - 1)}
    \end{align}
    The third term in Eq.(\ref{eq::matrix25}) corresponds to the quark-antiquark-gluon condensate.
    \begin{align}
        \bra{0}\bar{q}^{i}_{\alpha}(0) D_{\mu} D_{\nu} q^{j}_{\delta}(0) \ket{0}  &= C_{1} \delta^{ij} \delta_{\alpha \delta} \, g_{\mu \nu} + C_{2}  \delta^{ij} (\sigma_{\mu \nu})_{\delta \alpha},
        \label{eq::quark-antiquark-gluon1}
    \end{align}
    where we are only interested in the symmetric part, because we will multiply this expression by $x^{\mu} x^{\nu}$ which is symmetric. Now to compute $C_{1}$ we rewrite again Eq.(\ref{eq::quark-antiquark-gluon1}) using translation invariance as:
    \begin{align}
        \bra{0}\bar{q}^{i}_{\alpha}(0) D_{\mu} D_{\nu} q^{j}_{\delta}(0) \ket{0}   - \bra{0}\bar{q}^{i}_{\alpha}(0)  D_{\nu} D_{\mu} q^{j}_{\delta}(0) \ket{0}  =& \;  C_{2} \cdot \delta^{ij} \Big((\sigma_{\mu \nu})_{\delta \alpha} - (\sigma_{\nu \mu})_{\delta \alpha} \Big) \nonumber \\ =& \;  2 \cdot C_{2}  \delta^{ij} (\sigma_{\mu \nu})_{\delta \alpha}
    \end{align}
    Using the definition of the gluon field strength tensor $G_{\mu \nu} = \frac{i}{g_s} [D_{\mu},D_{\nu}]$, we obtain
    \begin{align}
	    \bra{0}\bar{q}^{i}_{\alpha}(0) (-i) g_s G_{\mu \nu}(0) q_{\delta}^j(0) \ket{0} =& \;  2 \cdot C_{2}  \delta^{ij} (\sigma_{\mu \nu})_{\delta \alpha} \nonumber \\ \Rightarrow C_2 =& \; -\frac{i}{2} \cdot E
    \end{align}
	The relation between $C_2$ and $C_1$ can be obtained by using the Dirac equation (in my notes).
	\begin{align}
		C_1 = \frac{d - 1}{2} \cdot E
	\end{align}
	So the final result for the third term is:
	\begin{align}
		\frac{x^{\mu} x^{\nu}}{2} \cdot \bra{0}\bar{q}^{i}_{\alpha}(0) D_{\mu} D_{\nu} q^{j}_{\delta}(0) \ket{0} = \frac{x^2}{2} \cdot C_1 \delta^{ij} \delta_{\alpha \delta} = \frac{x^2}{2} \delta^{ij} \delta_{\alpha \delta} \cdot \bra{0} \bar{q} g_s \sigma \cdot G q \ket{0} \cdot \frac{1}{8 N_c d}
	\end{align}
	With this expression it is possible to reproduce the desired expression.
    \end{document}