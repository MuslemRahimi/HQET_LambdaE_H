\documentclass[openright,twoside,12pt,a4paper,final]{article}
\usepackage{a4wide}
\usepackage[utf8]{inputenc}
\usepackage{blindtext}
\usepackage{amsmath}
\usepackage{amsfonts}
\usepackage{amssymb}
\usepackage{makeidx}
\usepackage{graphicx}
\usepackage{float}
\usepackage{titlesec}
\usepackage{listings}
\usepackage{hyperref}
\usepackage[english]{babel}
%\usepackage{classicthesis}
\usepackage{xcolor}
\usepackage{tikz}
\usepackage{pdfpages}
\usepackage[ngerman]{datetime}
\usepackage{bm}
\usepackage{subcaption}
\usepackage{changepage}
\usepackage{braket}
\usepackage{slashed}
\usepackage{simplewick}
%\usepackage[ngerman]{babel}
%\usepackage[T1]{fontenc}
%\usepackage{lmodern}

\allowdisplaybreaks

\begin{document}
	\noindent
	Here, we would like to derive the quark propagator in an external gluon field up to the first order correction. Starting point is the leading order contribution:
	\begin{align}
		S^{(0)} (x,0) =& \; i \int \frac{\mathrm{d}^dp}{(2 \pi)^d} e^{-ipx} \frac{\slashed{p}}{p^2 + i 0^+} \nonumber \\ =& \; i \int \frac{\mathrm{d}^d \vec{p}}{(2 \pi)^d} e^{i\vec{p} \vec{x}} \frac{- \slashed{\vec{p}}}{- \vec{p}^2 + i 0^+} \nonumber \\ =& - \slashed{\partial}_x \int \frac{\mathrm{d}^d \vec{p}}{(2 \pi)^d} e^{i\vec{p} \vec{x}} \frac{1}{- \vec{p}^2 + i 0^+} \nonumber \\ =& \; \frac{i}{4} \pi^{-\frac{d}{2}} \cdot \Gamma\Big[\frac{d}{2} - 1\Big] \Big(1 - \frac{d}{2}\Big) \frac{-2 \slashed{x}}{(-x^2 - i0^+)^{\frac{d}{2}}} \nonumber \\ =& \; i \frac{\Gamma\Big[\frac{d}{2}\Big]}{2 \pi^{\frac{d}{2}}} \frac{\slashed{x}}{(-x^2 - i0^+)^{\frac{d}{2}}}
	\end{align}
	\textcolor{red}{This proof turns out to be wrong. The paper by Grozin does not work in position space, as we used for our derivation. This is indeed not possible since we would have to be careful with the derivative and also with the $\slashed{x}$ in the numerator and in the calculation below (for example for the Wilson coefficients) everything would be suspicious.} \\
	Generally, Grozin is using a different notation in \cite{grozin2000lectures} for the propagator in momentum space. Additionally, he might also use a different notation for the Fourier transform of this propagator. Instead of eq. (2.10), the transformation should rather be of the following form:
	\begin{align}
		\int \frac{e^{-ipx}}{(-p^2 - i0^+)^n} \frac{\mathrm{d}^d p}{(2 \pi)^d} = \; i \cdot 2^{-2n} \pi^{-\frac{d}{2}} \frac{\Gamma(\frac{d}{2} - n)}{\Gamma(n)} \frac{1}{(-x^2 + i0^+)^{\frac{d}{2} - n}} \label{eq:Fourier}
	\end{align}
	The RHS of $\eqref{eq:Fourier}$ should correctly reproduce the scalar massless Feynman propagator ($n = 1$). This is in agreement with the results in \cite{HomeworkRastelli}, where the propagator has been determined with the residue theorem and for $d = 4$ both expressions agree. Another check provides the paper \cite{Hong_Hao_2010}, where complete analytic formulas for the massive Feynman propagator in d dimensions are stated. Taking the limit $m \rightarrow 0$ reproduces our result. The strongest check provides \cite{Novikov:1983gd}, where the Fourier transform from the x-space to the p-space is explicitly performed (check Appendix). The prefactors, which appear during the transformation, correspond to the statement in \eqref{eq:Fourier}. \\
	In order to obtain the massless fermion propagator, one has to take the derivative on the LHS and set $n = 1$:
	\begin{align}
		\int \frac{i\slashed{p}}{(-p^2 - i0^+)} e^{-ipx} \frac{\mathrm{d}^d p}{(2 \pi)^d} =& \; i^2 \slashed{\partial}_x \int \frac{1}{(p^2 + i0^+)} e^{-ipx} \frac{\mathrm{d}^d p}{(2 \pi)^d} \overset{\eqref{eq:Fourier}}{=} i \slashed{\partial}_x \cdot \Bigg[ 2^{2 - d} \pi^{-\frac{d}{2}} \frac{\Gamma(\frac{d}{2} - 1)}{\Gamma(1)} \frac{1}{(-x^2 + i0^+)^{\frac{d}{2} - 1}} \Bigg] \nonumber \\ =& \; i 2^{3 - d} \pi^{-\frac{d}{2}} \frac{\Gamma(\frac{d}{2})}{\Gamma(1)} \frac{\slashed{x}}{(-x^2 + i0^+)^{\frac{d}{2}}}
	\end{align} \\
	Now consider the correction to the quark propagator. This case is more involved, but the missing sign error might be cured. The definition is quite questionable, since \cite{Novikov:1983gd} and \cite{Gelhausen:2015qji} differ by a sign. Notice that \cite{Novikov:1983gd} performs all calculations in the Euclidean space, this might cause the sign difference. Therefore we stick to the definition in \cite{Gelhausen:2015qji}. Physically, this series expansion occurs in the case where the field strength of the gluon field $A^a_{\mu}(z)$ is small compared to the typically considered distances $|x - y|$. So the quark might scatter with gluon fields created from the vacuum.
	\begin{align}
		S(x,0) = S^{(0)}(x,0) \underbrace{-ig_s \int \mathrm{d}^d p S^{(0)}(x - p,0) \gamma^{\mu} T^a A^a_{\mu}(p) S^{(0)}(0,p)}_{S^{(1)}(x,0)}
	\end{align}
	\begin{align}
		S^{(1)}(x,0) =& \; -\frac{g_s}{2} \int \mathrm{d}^d p T^a G^a_{\alpha \mu}(0) \int \frac{\mathrm{d}^dk}{(2 \pi)^d} e^{-ik\cdot(x - p)} \frac{\slashed{k}}{k^2 + i0^+} \gamma^{\mu} \int \frac{\mathrm{d}^dk_1}{(2 \pi)^d} \frac{\slashed{k_1}}{k_1^2 + i0^+} \frac{\partial}{\partial k_{1,\alpha}} e^{-ik_1p} \nonumber \\ =& \;  \frac{g_s}{2} T^a G^a_{\alpha \mu}(0) \int \frac{\mathrm{d}^dk}{(2 \pi)^d} e^{-ikx} \frac{\slashed{k}}{k^2 + i0^+} \gamma^{\mu} \frac{\partial}{\partial k_{\alpha}} \frac{\slashed{k}}{k^2 + i0^+} \nonumber \\ =& \;  \frac{g_s}{2} T^a G^a_{\alpha \mu}(0) \int \frac{\mathrm{d}^dk}{(2 \pi)^d} e^{-ikx} \frac{\slashed{k}}{k^2 + i0^+} \gamma^{\mu} \Bigg[ \frac{\gamma^{\alpha}}{k^2 + i0^+} - \frac{\slashed{k}}{(k^2 + i0^+)^2} \cdot 2 k^{\alpha}\Bigg] \nonumber 
	\end{align}
	We now use the following relation between two gamma matrices:
	\begin{align}
		\gamma^{\mu} \gamma^{\alpha} = -i \sigma^{\mu \alpha} + g^{\mu \alpha}
	\end{align}
	Using this relation, we obtain:
	\begin{footnotesize}
	\begin{align}
		S^{(1)}(x,0) =& \;-\frac{g_s}{2} T^a G^a_{\alpha \mu}(0) \int \frac{\mathrm{d}^d k}{(2 \pi)^d} e^{-ikx} \frac{\slashed{k}}{k^2 + i0^+} \Bigg[ \frac{-i \sigma^{\mu \alpha} + g^{\mu \alpha}}{k^2 + i0^+} - \frac{4 k^{\mu} k^{\alpha}}{(k^2 + i0^+)^2} + \frac{2 \slashed{k} \gamma^{\mu} k^{\alpha}}{(k^2 + i0^+)^2} +  \frac{ \slashed{k} \gamma^{\alpha} k^{\mu}}{(k^2 + i0^+)^2} -  \\& \frac{ \slashed{k} \gamma^{\alpha} k^{\mu}}{(k^2 + i0^+)^2} \Bigg] \nonumber
	\end{align}
	\end{footnotesize} \normalsize
	\noindent
	Now we need to keep in mind that $S^{(1)}$ is invariant under the exchange of two Lorentz indices, while $G^a_{\alpha \mu}$ is not. Hence all symmetric terms can be dropped and only the antisymmetric terms survive:
	\begin{align}
		S^{(1)}(x,0) =& \; i \frac{g_s}{2} T^a G^a_{\alpha \mu}(0) \int \frac{\mathrm{d}^d k}{(2 \pi)^d} e^{-ikx} \Bigg[ \frac{\slashed{k} \sigma^{\mu \alpha}}{(k^2 + i0^+)^2} + i\frac{ \gamma^{\mu} k^{\alpha} - \gamma^{\alpha} k^{\mu}}{(k^2 + i0^+)^2} \Bigg] \nonumber \\ =& \;- i \frac{g_s}{2} T^a G^a_{\alpha \mu}(0) \int \frac{\mathrm{d}^d k}{(2 \pi)^d} e^{-ikx} \Bigg[ \frac{\slashed{k} \sigma^{ \alpha \mu}}{(k^2 + i0^+)^2} - i\frac{ \gamma^{\mu} k^{\alpha} - \gamma^{\alpha} k^{\mu}}{(k^2 + i0^+)^2} \Bigg] \nonumber \\ =& \; i \frac{g_s}{2} T^a G^a_{\alpha \mu}(0) \frac{1}{16 \pi^\frac{d}{2}} \Gamma\Big[\frac{d}{2} - 2\Big] \Big(2 - \frac{d}{2}\Big) \frac{1}{(-x^2 - i0^+)^{\frac{d}{2} - 1}} \Bigg[ -2\slashed{x} \sigma^{ \alpha \mu} - i\big( -2 \gamma^{\mu} x^{\alpha} +2 \gamma^{\alpha} x^{\mu}\big) \Bigg] \nonumber \\ =& \; i \frac{g_s}{2} T^a G^a_{\alpha \mu}(0) \frac{1}{16 \pi^\frac{d}{2}} \Gamma\Big[\frac{d}{2} - 1\Big]  \frac{1}{(-x^2 - i0^+)^{\frac{d}{2} - 1}} \underbrace{\Bigg[ 2\slashed{x} \sigma^{ \alpha \mu} + i \big(- 2 \gamma^{\mu} x^{\alpha} + 2 \gamma^{\alpha} x^{\mu}\big) \Bigg]}_{ = \{\slashed{x},\sigma^{\alpha \mu}\}} \nonumber
	\end{align}\\
	This corrections occurs in the derivation of the gluon-gluon condensate $<G^a_{\mu \nu} G^{a,\mu \nu}>$.
	The derivation of eq.(30) works in a similar way. Since only the first term in this expansion is of relevance for our LO + leading power corrections calculation, we will only demonstrate its computation and leave the other part for future work.
	\begin{align}
		\bra{0} \contraction{}{G^a_{\mu \nu}(0)}{ }{A^d_{\lambda}(z)} G^a_{\mu \nu}(0) A^d_{\lambda}(z) \ket{0} =& \bra{0} \contraction{}{\partial_{\mu} A^a_{\nu}}{-\partial_{\nu} A^a_{\mu} + g_s f^{abc} A^b_{\mu} A^c_{\nu} }{A^d_{\lambda}(z)} (\partial_{\mu} A^a_{\nu} -\partial_{\nu} A^a_{\mu} + g_s f^{abc} A^b_{\mu} A^c_{\nu}) A^d_{\lambda}(z) \ket{0} + \bra{0} \contraction{\partial_{\mu} A^a_{\nu}}{ -\partial_{\nu} A^a_{\mu}}{ + g_s f^{abc} A^b_{\mu} A^c_{\nu} }{A^d_{\lambda}(z)} (\partial_{\mu} A^a_{\nu} -\partial_{\nu} A^a_{\mu} + g_s f^{abc} A^b_{\mu} A^c_{\nu}) A^d_{\lambda}(z) \ket{0} \nonumber \\ =& \partial_{\mu} \Bigg[ \int \frac{\mathrm{d}^d k}{(2 \Pi)^d} e^{-ik(x - z)} \frac{-i g_{\nu \lambda}}{k^2 + i0^+} \left. \delta^{ad}\Bigg] \right|_{x = 0} - \partial_{\nu} \Bigg[\int \frac{\mathrm{d}^d k}{(2 \Pi)^d} e^{-ik(x - z)} \frac{-i g_{\mu \lambda}}{k^2 + i0^+} \left. \delta^{ad}\Bigg] \right|_{x = 0} \nonumber \\ =& -i g_{\nu \lambda} \delta^{ad} \partial_{\mu} \Bigg[-i \frac{1}{4 \pi^{\frac{d}{2}}} \Gamma\Big(\frac{d}{2} - 1\Big) \frac{1}{(-(x - z)^2 + i0^+)^{\frac{d}{2} - 1}}\left.\Bigg] \right|_{x=0} + \nonumber \\& i g_{\mu \lambda} \delta^{ad} \partial_{\nu} \Bigg[-i \frac{1}{4 \pi^{\frac{d}{2}}} \Gamma\Big(\frac{d}{2} - 1\Big) \frac{1}{(-(x - z)^2 + i0^+)^{\frac{d}{2} - 1}}\left.\Bigg] \right|_{x=0} \nonumber \\ =&  g_{\nu \lambda} \delta^{ad} \frac{\Gamma\Big(\frac{d}{2}\Big)}{4 \pi^{\frac{d}{2}}}  \frac{2 z_{\mu}}{(-z^2 + i0^+)^{\frac{d}{2} - 1}}\left.\Bigg] \right|_{x=0}  \nonumber - g_{\mu \lambda} \delta^{ad} \frac{\Gamma\Big(\frac{d}{2}\Big)}{4 \pi^{\frac{d}{2}}} \frac{2 z_{\nu }}{(-z^2 + i0^+)^{\frac{d}{2} - 1}}\left.\Bigg] \right|_{x=0}
	\end{align}
	
	
	\bibliographystyle{unsrt}
	\bibliography{ReferencesPropPosSpace}
\end{document}